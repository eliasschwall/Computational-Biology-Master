% Options for packages loaded elsewhere
\PassOptionsToPackage{unicode}{hyperref}
\PassOptionsToPackage{hyphens}{url}
%
\documentclass[
]{article}
\usepackage{amsmath,amssymb}
\usepackage{iftex}
\ifPDFTeX
  \usepackage[T1]{fontenc}
  \usepackage[utf8]{inputenc}
  \usepackage{textcomp} % provide euro and other symbols
\else % if luatex or xetex
  \usepackage{unicode-math} % this also loads fontspec
  \defaultfontfeatures{Scale=MatchLowercase}
  \defaultfontfeatures[\rmfamily]{Ligatures=TeX,Scale=1}
\fi
\usepackage{lmodern}
\ifPDFTeX\else
  % xetex/luatex font selection
\fi
% Use upquote if available, for straight quotes in verbatim environments
\IfFileExists{upquote.sty}{\usepackage{upquote}}{}
\IfFileExists{microtype.sty}{% use microtype if available
  \usepackage[]{microtype}
  \UseMicrotypeSet[protrusion]{basicmath} % disable protrusion for tt fonts
}{}
\makeatletter
\@ifundefined{KOMAClassName}{% if non-KOMA class
  \IfFileExists{parskip.sty}{%
    \usepackage{parskip}
  }{% else
    \setlength{\parindent}{0pt}
    \setlength{\parskip}{6pt plus 2pt minus 1pt}}
}{% if KOMA class
  \KOMAoptions{parskip=half}}
\makeatother
\usepackage{xcolor}
\usepackage[margin=1in]{geometry}
\usepackage{color}
\usepackage{fancyvrb}
\newcommand{\VerbBar}{|}
\newcommand{\VERB}{\Verb[commandchars=\\\{\}]}
\DefineVerbatimEnvironment{Highlighting}{Verbatim}{commandchars=\\\{\}}
% Add ',fontsize=\small' for more characters per line
\usepackage{framed}
\definecolor{shadecolor}{RGB}{248,248,248}
\newenvironment{Shaded}{\begin{snugshade}}{\end{snugshade}}
\newcommand{\AlertTok}[1]{\textcolor[rgb]{0.94,0.16,0.16}{#1}}
\newcommand{\AnnotationTok}[1]{\textcolor[rgb]{0.56,0.35,0.01}{\textbf{\textit{#1}}}}
\newcommand{\AttributeTok}[1]{\textcolor[rgb]{0.13,0.29,0.53}{#1}}
\newcommand{\BaseNTok}[1]{\textcolor[rgb]{0.00,0.00,0.81}{#1}}
\newcommand{\BuiltInTok}[1]{#1}
\newcommand{\CharTok}[1]{\textcolor[rgb]{0.31,0.60,0.02}{#1}}
\newcommand{\CommentTok}[1]{\textcolor[rgb]{0.56,0.35,0.01}{\textit{#1}}}
\newcommand{\CommentVarTok}[1]{\textcolor[rgb]{0.56,0.35,0.01}{\textbf{\textit{#1}}}}
\newcommand{\ConstantTok}[1]{\textcolor[rgb]{0.56,0.35,0.01}{#1}}
\newcommand{\ControlFlowTok}[1]{\textcolor[rgb]{0.13,0.29,0.53}{\textbf{#1}}}
\newcommand{\DataTypeTok}[1]{\textcolor[rgb]{0.13,0.29,0.53}{#1}}
\newcommand{\DecValTok}[1]{\textcolor[rgb]{0.00,0.00,0.81}{#1}}
\newcommand{\DocumentationTok}[1]{\textcolor[rgb]{0.56,0.35,0.01}{\textbf{\textit{#1}}}}
\newcommand{\ErrorTok}[1]{\textcolor[rgb]{0.64,0.00,0.00}{\textbf{#1}}}
\newcommand{\ExtensionTok}[1]{#1}
\newcommand{\FloatTok}[1]{\textcolor[rgb]{0.00,0.00,0.81}{#1}}
\newcommand{\FunctionTok}[1]{\textcolor[rgb]{0.13,0.29,0.53}{\textbf{#1}}}
\newcommand{\ImportTok}[1]{#1}
\newcommand{\InformationTok}[1]{\textcolor[rgb]{0.56,0.35,0.01}{\textbf{\textit{#1}}}}
\newcommand{\KeywordTok}[1]{\textcolor[rgb]{0.13,0.29,0.53}{\textbf{#1}}}
\newcommand{\NormalTok}[1]{#1}
\newcommand{\OperatorTok}[1]{\textcolor[rgb]{0.81,0.36,0.00}{\textbf{#1}}}
\newcommand{\OtherTok}[1]{\textcolor[rgb]{0.56,0.35,0.01}{#1}}
\newcommand{\PreprocessorTok}[1]{\textcolor[rgb]{0.56,0.35,0.01}{\textit{#1}}}
\newcommand{\RegionMarkerTok}[1]{#1}
\newcommand{\SpecialCharTok}[1]{\textcolor[rgb]{0.81,0.36,0.00}{\textbf{#1}}}
\newcommand{\SpecialStringTok}[1]{\textcolor[rgb]{0.31,0.60,0.02}{#1}}
\newcommand{\StringTok}[1]{\textcolor[rgb]{0.31,0.60,0.02}{#1}}
\newcommand{\VariableTok}[1]{\textcolor[rgb]{0.00,0.00,0.00}{#1}}
\newcommand{\VerbatimStringTok}[1]{\textcolor[rgb]{0.31,0.60,0.02}{#1}}
\newcommand{\WarningTok}[1]{\textcolor[rgb]{0.56,0.35,0.01}{\textbf{\textit{#1}}}}
\usepackage{graphicx}
\makeatletter
\def\maxwidth{\ifdim\Gin@nat@width>\linewidth\linewidth\else\Gin@nat@width\fi}
\def\maxheight{\ifdim\Gin@nat@height>\textheight\textheight\else\Gin@nat@height\fi}
\makeatother
% Scale images if necessary, so that they will not overflow the page
% margins by default, and it is still possible to overwrite the defaults
% using explicit options in \includegraphics[width, height, ...]{}
\setkeys{Gin}{width=\maxwidth,height=\maxheight,keepaspectratio}
% Set default figure placement to htbp
\makeatletter
\def\fps@figure{htbp}
\makeatother
\setlength{\emergencystretch}{3em} % prevent overfull lines
\providecommand{\tightlist}{%
  \setlength{\itemsep}{0pt}\setlength{\parskip}{0pt}}
\setcounter{secnumdepth}{-\maxdimen} % remove section numbering
\ifLuaTeX
  \usepackage{selnolig}  % disable illegal ligatures
\fi
\IfFileExists{bookmark.sty}{\usepackage{bookmark}}{\usepackage{hyperref}}
\IfFileExists{xurl.sty}{\usepackage{xurl}}{} % add URL line breaks if available
\urlstyle{same}
\hypersetup{
  pdftitle={CompBio Homework SCHWALL ELIAS},
  pdfauthor={Elias Schwall},
  hidelinks,
  pdfcreator={LaTeX via pandoc}}

\title{CompBio Homework SCHWALL ELIAS}
\author{Elias Schwall}
\date{2024-01-21}

\begin{document}
\maketitle

\hypertarget{task-1-pizza-and-markov-chains}{%
\section{Task 1 Pizza and Markov
chains}\label{task-1-pizza-and-markov-chains}}

\hypertarget{a-give-the-transition-matrix-to-this-process}{%
\subsection{a) Give the transition matrix to this
process}\label{a-give-the-transition-matrix-to-this-process}}

\begin{Shaded}
\begin{Highlighting}[]
\NormalTok{transition }\OtherTok{\textless{}{-}} \FunctionTok{matrix}\NormalTok{(}\FunctionTok{c}\NormalTok{(}\FloatTok{0.4}\NormalTok{,}\FloatTok{0.3}\NormalTok{,}\FloatTok{0.3}\NormalTok{,}
                       \FloatTok{0.0}\NormalTok{,}\FloatTok{0.1}\NormalTok{,}\FloatTok{0.9}\NormalTok{,}
                       \FloatTok{0.1}\NormalTok{,}\FloatTok{0.4}\NormalTok{,}\FloatTok{0.5}\NormalTok{), }\AttributeTok{nrow =} \DecValTok{3}\NormalTok{, }\AttributeTok{ncol =} \DecValTok{3}\NormalTok{, }\AttributeTok{byrow =} \ConstantTok{TRUE}\NormalTok{)}

\NormalTok{names }\OtherTok{\textless{}{-}} \FunctionTok{c}\NormalTok{(}\StringTok{"Aurora Pizza"}\NormalTok{, }\StringTok{"Pizza Boy"}\NormalTok{, }\StringTok{"Pizza Company"}\NormalTok{)}
\FunctionTok{rownames}\NormalTok{(transition) }\OtherTok{\textless{}{-}}\NormalTok{ names}
\FunctionTok{colnames}\NormalTok{(transition) }\OtherTok{\textless{}{-}}\NormalTok{ names}
\FunctionTok{print}\NormalTok{(transition)}
\end{Highlighting}
\end{Shaded}

\begin{verbatim}
##               Aurora Pizza Pizza Boy Pizza Company
## Aurora Pizza           0.4       0.3           0.3
## Pizza Boy              0.0       0.1           0.9
## Pizza Company          0.1       0.4           0.5
\end{verbatim}

\hypertarget{b-calculate-the-costumer-distribution-after-1210-and-20-month}{%
\subsection{b) Calculate the costumer distribution after 1,2,10 and 20
month}\label{b-calculate-the-costumer-distribution-after-1210-and-20-month}}

\begin{Shaded}
\begin{Highlighting}[]
\NormalTok{time\_frames }\OtherTok{\textless{}{-}} \FunctionTok{c}\NormalTok{(}\DecValTok{1}\NormalTok{,}\DecValTok{2}\NormalTok{,}\DecValTok{10}\NormalTok{,}\DecValTok{20}\NormalTok{)}

\ControlFlowTok{for}\NormalTok{ (i }\ControlFlowTok{in}\NormalTok{ time\_frames)\{}
\NormalTok{    customer\_distribution }\OtherTok{\textless{}{-}} \FunctionTok{c}\NormalTok{(}\DecValTok{200}\NormalTok{,}\DecValTok{300}\NormalTok{,}\DecValTok{100}\NormalTok{)}
    \ControlFlowTok{for}\NormalTok{ (j }\ControlFlowTok{in} \DecValTok{1}\SpecialCharTok{:}\NormalTok{i) \{}
\NormalTok{        customer\_distribution }\OtherTok{\textless{}{-}}\NormalTok{ customer\_distribution }\SpecialCharTok{\%*\%}\NormalTok{ transition}
\NormalTok{    \}}
    \FunctionTok{cat}\NormalTok{(}\StringTok{"Customer distrinbution after"}\NormalTok{, i, }\StringTok{"months"}\NormalTok{,customer\_distribution,}\StringTok{"}\SpecialCharTok{\textbackslash{}n}\StringTok{"}\NormalTok{, }\AttributeTok{sep =} \StringTok{" "}\NormalTok{)}
\NormalTok{\}}
\end{Highlighting}
\end{Shaded}

\begin{verbatim}
## Customer distrinbution after 1 months 90 130 380 
## Customer distrinbution after 2 months 74 192 334 
## Customer distrinbution after 10 months 60.0014 180.0012 359.9974 
## Customer distrinbution after 20 months 60 180 360
\end{verbatim}

\hypertarget{c-interpret-the-result-using-the-term-stationary-distribution}{%
\subsection{c) Interpret the result, using the term 'stationary
distribution'}\label{c-interpret-the-result-using-the-term-stationary-distribution}}

After 10 months we can see that the customer distribution is basically
not changing anymore (rounded). This is called the stationary
distribution. The stationary distribution is the long term distribution
of the customer distribution. This implies that following a specific
duration, the customer distribution for each pizza store will reach a
stable state, and the market shares will remain relatively consistent
over time. In principle the stantionary distribution is achieved earlier
than after 10 months, however in the observed time frames we see
behavior after 10 months, when comparing it to 20.

\hypertarget{task-2-asymmetric-random-walk---an-urn-model}{%
\section{Task 2 Asymmetric random walk - an urn
model}\label{task-2-asymmetric-random-walk---an-urn-model}}

\hypertarget{a-define-the-procedure-draw-and-replace-in-the-r-script-which-updates-the-balls-in-the-urn.-for-weighted-sampling-use-sample-with-the-option-prob-.-the-vector-can-be-defined-with-the-weights-1-and-s-1-where-balls0-and-s-where-balls1}{%
\subsection{a) Define the procedure draw and replace in the R-script,
which updates the balls in the urn. For weighted sampling, use sample
with the option prob= \ldots. The vector can be defined with the weights
1 and s (1 where balls=0 and s where
balls=1)}\label{a-define-the-procedure-draw-and-replace-in-the-r-script-which-updates-the-balls-in-the-urn.-for-weighted-sampling-use-sample-with-the-option-prob-.-the-vector-can-be-defined-with-the-weights-1-and-s-1-where-balls0-and-s-where-balls1}}

\begin{Shaded}
\begin{Highlighting}[]
\CommentTok{\#Urn{-}model\#}
\NormalTok{N }\OtherTok{\textless{}{-}} \DecValTok{100} \CommentTok{\#number of balls}
\NormalTok{s }\OtherTok{\textless{}{-}} \DecValTok{2} \CommentTok{\#weight of white balls}

\NormalTok{balls }\OtherTok{\textless{}{-}} \FunctionTok{rep}\NormalTok{(}\DecValTok{0}\NormalTok{,N)}
\NormalTok{balls[}\DecValTok{1}\NormalTok{] }\OtherTok{\textless{}{-}} \DecValTok{1} \CommentTok{\#start with one white ball}

\DocumentationTok{\#\#\#\#\#\#\#\#\#\#\#\#\#\#\#\#\#\#\#\#\#\#\#\#\#\#\#\#\#\#\#\#\#\#\#\#\#\#\#\#}
\CommentTok{\# Define the function draw and replace \#}
\DocumentationTok{\#\#\#\#\#\#\#\#\#\#\#\#\#\#\#\#\#\#\#\#\#\#\#\#\#\#\#\#\#\#\#\#\#\#\#\#\#\#\#\#}

\NormalTok{draw\_and\_replace }\OtherTok{\textless{}{-}} \ControlFlowTok{function}\NormalTok{(balls, }\AttributeTok{s=}\DecValTok{2}\NormalTok{)\{}
  \CommentTok{\# Random deletion}
\NormalTok{  balls }\OtherTok{\textless{}{-}}\NormalTok{ balls[}\SpecialCharTok{{-}}\FunctionTok{sample}\NormalTok{(}\FunctionTok{length}\NormalTok{(balls),}\DecValTok{1}\NormalTok{)]}
  \CommentTok{\# Weighted duplication}
\NormalTok{  prob\_weights }\OtherTok{\textless{}{-}}\NormalTok{ balls}
\NormalTok{  prob\_weights[prob\_weights }\SpecialCharTok{==} \DecValTok{1}\NormalTok{] }\OtherTok{\textless{}{-}}\NormalTok{ s}
\NormalTok{  prob\_weights[prob\_weights }\SpecialCharTok{==} \DecValTok{0}\NormalTok{] }\OtherTok{\textless{}{-}} \DecValTok{1}
\NormalTok{  selected\_ball }\OtherTok{\textless{}{-}} \FunctionTok{sample}\NormalTok{(balls, }\DecValTok{1}\NormalTok{, }\AttributeTok{prob =}\NormalTok{ prob\_weights)}
\NormalTok{  balls }\OtherTok{\textless{}{-}} \FunctionTok{c}\NormalTok{(balls, selected\_ball)}
  \FunctionTok{return}\NormalTok{(balls)}
\NormalTok{\}}

\DocumentationTok{\#\#\# Try out }\AlertTok{\#\#\#}
\NormalTok{balls }\OtherTok{\textless{}{-}} \FunctionTok{draw\_and\_replace}\NormalTok{(balls, }\AttributeTok{s=}\DecValTok{2}\NormalTok{)  }
\FunctionTok{print}\NormalTok{(balls)}
\end{Highlighting}
\end{Shaded}

\begin{verbatim}
##   [1] 1 0 0 0 0 0 0 0 0 0 0 0 0 0 0 0 0 0 0 0 0 0 0 0 0 0 0 0 0 0 0 0 0 0 0 0 0
##  [38] 0 0 0 0 0 0 0 0 0 0 0 0 0 0 0 0 0 0 0 0 0 0 0 0 0 0 0 0 0 0 0 0 0 0 0 0 0
##  [75] 0 0 0 0 0 0 0 0 0 0 0 0 0 0 0 0 0 0 0 0 0 0 0 0 0 0
\end{verbatim}

\hypertarget{b-describe-the-purpose-of-the-function-draw-trajectory.-choose-at-least-3-different-parameter-combinations-varying-s-n-run-time-etc.-and-attach-the-plots-to-the-pdf.-give-a-short-interpretation-of-the-effects-of-the-parameters.}{%
\subsection{b) Describe the purpose of the function draw trajectory.
Choose at least 3 different parameter combinations (varying s, N, run
time etc.) and attach the plots to the pdf. Give a short interpretation
of the effects of the
parameters.}\label{b-describe-the-purpose-of-the-function-draw-trajectory.-choose-at-least-3-different-parameter-combinations-varying-s-n-run-time-etc.-and-attach-the-plots-to-the-pdf.-give-a-short-interpretation-of-the-effects-of-the-parameters.}}

The draw\_trajectory function is used to simulate the urn model and plot
the trajectory of the number of balls in the urn over time. The function
takes the following parameters: repl is the number of simulations to be
run, run\_time is the number of time steps to be simulated, s is the
weight of the white balls, and N is the number of balls in the urn. The
function is used to investigate the effect of the parameters on the urn
model. The following plots show the effect of varying s, N, and repl on
the urn model.

\begin{Shaded}
\begin{Highlighting}[]
\NormalTok{draw\_trajectory }\OtherTok{\textless{}{-}} \ControlFlowTok{function}\NormalTok{(}\AttributeTok{repl=}\DecValTok{20}\NormalTok{, }\AttributeTok{run\_time=}\FloatTok{1e3}\NormalTok{, }\AttributeTok{s=}\DecValTok{2}\NormalTok{, }\AttributeTok{N=}\DecValTok{100}\NormalTok{)\{}
\NormalTok{  s }\OtherTok{\textless{}{-}}\NormalTok{ s}
\NormalTok{  trajectory }\OtherTok{\textless{}{-}} \FunctionTok{rep}\NormalTok{(}\DecValTok{0}\NormalTok{,run\_time)}
  \FunctionTok{plot}\NormalTok{(}\DecValTok{0}\NormalTok{,}\DecValTok{0}\NormalTok{,}\AttributeTok{xlim=}\FunctionTok{c}\NormalTok{(}\DecValTok{0}\NormalTok{,run\_time),}\AttributeTok{ylim=}\FunctionTok{c}\NormalTok{(}\DecValTok{0}\NormalTok{,N),}\AttributeTok{xlab=}\StringTok{"time"}\NormalTok{,}\AttributeTok{ylab=}\StringTok{"number"}\NormalTok{,}\AttributeTok{main=}\FunctionTok{paste0}\NormalTok{(}\StringTok{"s= "}\NormalTok{,s,}\StringTok{", N="}\NormalTok{,N,}\StringTok{", repl= "}\NormalTok{,repl,}\StringTok{", run\_time="}\NormalTok{,run\_time,}\AttributeTok{sep=}\StringTok{""}\NormalTok{))}
  \ControlFlowTok{for}\NormalTok{ (it }\ControlFlowTok{in} \DecValTok{1}\SpecialCharTok{:}\NormalTok{repl) \{}
\NormalTok{    balls }\OtherTok{\textless{}{-}} \FunctionTok{rep}\NormalTok{(}\DecValTok{0}\NormalTok{,N)}
\NormalTok{    balls[}\DecValTok{1}\NormalTok{] }\OtherTok{\textless{}{-}} \DecValTok{1}
    
    \ControlFlowTok{for}\NormalTok{ (tm }\ControlFlowTok{in} \DecValTok{1}\SpecialCharTok{:}\NormalTok{run\_time) \{}
\NormalTok{      trajectory[tm] }\OtherTok{\textless{}{-}} \FunctionTok{sum}\NormalTok{(balls)}
\NormalTok{      balls }\OtherTok{\textless{}{-}} \FunctionTok{draw\_and\_replace}\NormalTok{(balls, s)}
\NormalTok{    \}}
    
    \FunctionTok{points}\NormalTok{(}\DecValTok{1}\SpecialCharTok{:}\NormalTok{run\_time,trajectory,}\AttributeTok{type=}\StringTok{"l"}\NormalTok{)}
\NormalTok{  \}}
\NormalTok{\}}
\end{Highlighting}
\end{Shaded}

\hypertarget{plots-for-varying-s}{%
\subsubsection{Plots for varying s}\label{plots-for-varying-s}}

\begin{Shaded}
\begin{Highlighting}[]
\CommentTok{\# Changing s }
\FunctionTok{draw\_trajectory}\NormalTok{(}\AttributeTok{repl=}\DecValTok{20}\NormalTok{, }\AttributeTok{run\_time=}\FloatTok{1e3}\NormalTok{, }\AttributeTok{s=}\DecValTok{1}\NormalTok{, }\AttributeTok{N=}\DecValTok{100}\NormalTok{)}
\end{Highlighting}
\end{Shaded}

\includegraphics{CompBio_Homework_SCHWALL_ELIAS_files/figure-latex/unnamed-chunk-5-1.pdf}

\begin{Shaded}
\begin{Highlighting}[]
\FunctionTok{draw\_trajectory}\NormalTok{(}\AttributeTok{repl=}\DecValTok{20}\NormalTok{, }\AttributeTok{run\_time=}\FloatTok{1e3}\NormalTok{, }\AttributeTok{s=}\DecValTok{5}\NormalTok{, }\AttributeTok{N=}\DecValTok{100}\NormalTok{)}
\end{Highlighting}
\end{Shaded}

\includegraphics{CompBio_Homework_SCHWALL_ELIAS_files/figure-latex/unnamed-chunk-5-2.pdf}

\begin{Shaded}
\begin{Highlighting}[]
\FunctionTok{draw\_trajectory}\NormalTok{(}\AttributeTok{repl=}\DecValTok{20}\NormalTok{, }\AttributeTok{run\_time=}\FloatTok{1e3}\NormalTok{, }\AttributeTok{s=}\DecValTok{10}\NormalTok{, }\AttributeTok{N=}\DecValTok{100}\NormalTok{)}
\end{Highlighting}
\end{Shaded}

\includegraphics{CompBio_Homework_SCHWALL_ELIAS_files/figure-latex/unnamed-chunk-5-3.pdf}
When changing the weight of the white balls, we can see that the number
of white balls in the urn increases faster when the weight is higher.
This is because the probability of drawing a white ball is higher when
the weight is higher. \newpage

\hypertarget{plots-for-varying-repl}{%
\subsubsection{Plots for varying repl}\label{plots-for-varying-repl}}

\begin{Shaded}
\begin{Highlighting}[]
\CommentTok{\# Changing repl}
\FunctionTok{draw\_trajectory}\NormalTok{(}\AttributeTok{repl=}\DecValTok{20}\NormalTok{, }\AttributeTok{run\_time=}\FloatTok{1e3}\NormalTok{, }\AttributeTok{s=}\DecValTok{2}\NormalTok{, }\AttributeTok{N=}\DecValTok{100}\NormalTok{)}
\end{Highlighting}
\end{Shaded}

\includegraphics{CompBio_Homework_SCHWALL_ELIAS_files/figure-latex/unnamed-chunk-6-1.pdf}

\begin{Shaded}
\begin{Highlighting}[]
\FunctionTok{draw\_trajectory}\NormalTok{(}\AttributeTok{repl=}\DecValTok{30}\NormalTok{, }\AttributeTok{run\_time=}\FloatTok{1e3}\NormalTok{, }\AttributeTok{s=}\DecValTok{2}\NormalTok{, }\AttributeTok{N=}\DecValTok{100}\NormalTok{)}
\end{Highlighting}
\end{Shaded}

\includegraphics{CompBio_Homework_SCHWALL_ELIAS_files/figure-latex/unnamed-chunk-6-2.pdf}

\begin{Shaded}
\begin{Highlighting}[]
\FunctionTok{draw\_trajectory}\NormalTok{(}\AttributeTok{repl=}\DecValTok{50}\NormalTok{, }\AttributeTok{run\_time=}\FloatTok{1e3}\NormalTok{, }\AttributeTok{s=}\DecValTok{2}\NormalTok{, }\AttributeTok{N=}\DecValTok{100}\NormalTok{)}
\end{Highlighting}
\end{Shaded}

\includegraphics{CompBio_Homework_SCHWALL_ELIAS_files/figure-latex/unnamed-chunk-6-3.pdf}
When changing the number of simulations, we can see that there are more
trajectories. This is because the number of simulations is the number of
times the urn model is simulated. Therefore, it more likely that some
trajectories will approach the maximum number of white balls in the urn
by chance. \newpage

\hypertarget{plots-for-varying-n}{%
\subsubsection{Plots for varying N}\label{plots-for-varying-n}}

\begin{Shaded}
\begin{Highlighting}[]
\CommentTok{\# Changing N}
\FunctionTok{draw\_trajectory}\NormalTok{(}\AttributeTok{repl=}\DecValTok{20}\NormalTok{, }\AttributeTok{run\_time=}\FloatTok{1e3}\NormalTok{, }\AttributeTok{s=}\DecValTok{2}\NormalTok{, }\AttributeTok{N=}\DecValTok{100}\NormalTok{)}
\end{Highlighting}
\end{Shaded}

\includegraphics{CompBio_Homework_SCHWALL_ELIAS_files/figure-latex/unnamed-chunk-7-1.pdf}

\begin{Shaded}
\begin{Highlighting}[]
\FunctionTok{draw\_trajectory}\NormalTok{(}\AttributeTok{repl=}\DecValTok{20}\NormalTok{, }\AttributeTok{run\_time=}\FloatTok{1e3}\NormalTok{, }\AttributeTok{s=}\DecValTok{2}\NormalTok{, }\AttributeTok{N=}\DecValTok{500}\NormalTok{)}
\end{Highlighting}
\end{Shaded}

\includegraphics{CompBio_Homework_SCHWALL_ELIAS_files/figure-latex/unnamed-chunk-7-2.pdf}

\begin{Shaded}
\begin{Highlighting}[]
\FunctionTok{draw\_trajectory}\NormalTok{(}\AttributeTok{repl=}\DecValTok{20}\NormalTok{, }\AttributeTok{run\_time=}\FloatTok{1e3}\NormalTok{, }\AttributeTok{s=}\DecValTok{2}\NormalTok{, }\AttributeTok{N=}\DecValTok{1000}\NormalTok{)}
\end{Highlighting}
\end{Shaded}

\includegraphics{CompBio_Homework_SCHWALL_ELIAS_files/figure-latex/unnamed-chunk-7-3.pdf}
When changing the number of balls in the urn, we can see that it is more
likely to pick a black ball, because the ratio of white to black balls
changes (we still start only with one white ball). Therefore, it is more
likely that the trajectory will approach the minimum number of white
balls (0) in the urn by chance. \newpage

\hypertarget{plots-for-varying-run_time}{%
\subsubsection{Plots for varying
run\_time}\label{plots-for-varying-run_time}}

\begin{Shaded}
\begin{Highlighting}[]
\CommentTok{\# Changing run\_time}
\FunctionTok{draw\_trajectory}\NormalTok{(}\AttributeTok{repl=}\DecValTok{20}\NormalTok{, }\AttributeTok{run\_time=}\FloatTok{1e2}\NormalTok{, }\AttributeTok{s=}\DecValTok{2}\NormalTok{, }\AttributeTok{N=}\DecValTok{100}\NormalTok{)}
\end{Highlighting}
\end{Shaded}

\includegraphics{CompBio_Homework_SCHWALL_ELIAS_files/figure-latex/unnamed-chunk-8-1.pdf}

\begin{Shaded}
\begin{Highlighting}[]
\FunctionTok{draw\_trajectory}\NormalTok{(}\AttributeTok{repl=}\DecValTok{20}\NormalTok{, }\AttributeTok{run\_time=}\FloatTok{1e3}\NormalTok{, }\AttributeTok{s=}\DecValTok{2}\NormalTok{, }\AttributeTok{N=}\DecValTok{100}\NormalTok{)}
\end{Highlighting}
\end{Shaded}

\includegraphics{CompBio_Homework_SCHWALL_ELIAS_files/figure-latex/unnamed-chunk-8-2.pdf}

\begin{Shaded}
\begin{Highlighting}[]
\FunctionTok{draw\_trajectory}\NormalTok{(}\AttributeTok{repl=}\DecValTok{20}\NormalTok{, }\AttributeTok{run\_time=}\FloatTok{1e4}\NormalTok{, }\AttributeTok{s=}\DecValTok{2}\NormalTok{, }\AttributeTok{N=}\DecValTok{100}\NormalTok{)}
\end{Highlighting}
\end{Shaded}

\includegraphics{CompBio_Homework_SCHWALL_ELIAS_files/figure-latex/unnamed-chunk-8-3.pdf}
When changing the number of time steps, we can see that reaching a
minimum or maximum number of white balls in the urn becomes less likely,
due to less drawing and replacing steps.

\hypertarget{c-describe-the-purpose-of-the-function-get-trans-prob.-choose-at-least-10-different-parameter-combinations-varying-x0-s-n.-give-a-short-interpretation-of-your-result.}{%
\subsection{c) Describe the purpose of the function get trans prob.
Choose at least 10 different parameter combinations (varying x0, s, N).
Give a short interpretation of your
result.}\label{c-describe-the-purpose-of-the-function-get-trans-prob.-choose-at-least-10-different-parameter-combinations-varying-x0-s-n.-give-a-short-interpretation-of-your-result.}}

\begin{Shaded}
\begin{Highlighting}[]
\NormalTok{get\_trans\_prob }\OtherTok{\textless{}{-}} \ControlFlowTok{function}\NormalTok{(x0,N,s,}\AttributeTok{it=}\DecValTok{1000}\NormalTok{)\{}
\NormalTok{  x\_out }\OtherTok{\textless{}{-}} \FunctionTok{rep}\NormalTok{(}\DecValTok{0}\NormalTok{,it)}
  \ControlFlowTok{for}\NormalTok{(tm }\ControlFlowTok{in} \DecValTok{1}\SpecialCharTok{:}\NormalTok{it)\{}
\NormalTok{    x }\OtherTok{\textless{}{-}}\NormalTok{ x0}
\NormalTok{    y }\OtherTok{\textless{}{-}} \FunctionTok{sample}\NormalTok{(}\FunctionTok{c}\NormalTok{(}\DecValTok{0}\NormalTok{,}\DecValTok{1}\NormalTok{),}\DecValTok{1}\NormalTok{,}\AttributeTok{prob=}\FunctionTok{c}\NormalTok{(}\DecValTok{1}\SpecialCharTok{{-}}\NormalTok{x}\SpecialCharTok{/}\NormalTok{N,x}\SpecialCharTok{/}\NormalTok{N))}
    \ControlFlowTok{if}\NormalTok{(y}\SpecialCharTok{==}\DecValTok{1}\NormalTok{)\{}
\NormalTok{      x }\OtherTok{\textless{}{-}}\NormalTok{ x}\DecValTok{{-}1}
\NormalTok{    \}}
    
\NormalTok{    z }\OtherTok{\textless{}{-}} \FunctionTok{sample}\NormalTok{(}\FunctionTok{c}\NormalTok{(}\DecValTok{0}\NormalTok{,}\DecValTok{1}\NormalTok{),}\DecValTok{1}\NormalTok{,}\AttributeTok{prob=}\FunctionTok{c}\NormalTok{(}\DecValTok{1}\SpecialCharTok{{-}}\NormalTok{(s}\SpecialCharTok{*}\NormalTok{x)}\SpecialCharTok{/}\NormalTok{(s}\SpecialCharTok{*}\NormalTok{x}\SpecialCharTok{+}\NormalTok{(N}\DecValTok{{-}1}\NormalTok{)), (s}\SpecialCharTok{*}\NormalTok{x)}\SpecialCharTok{/}\NormalTok{(s}\SpecialCharTok{*}\NormalTok{x}\SpecialCharTok{+}\NormalTok{(N}\DecValTok{{-}1}\NormalTok{)) ))}
    \ControlFlowTok{if}\NormalTok{(z}\SpecialCharTok{==}\DecValTok{1}\NormalTok{)\{}
\NormalTok{      x }\OtherTok{\textless{}{-}}\NormalTok{ x}\SpecialCharTok{+}\DecValTok{1}
\NormalTok{    \}}
\NormalTok{    x\_out[tm] }\OtherTok{\textless{}{-}}\NormalTok{ x}
\NormalTok{  \}}
  \FunctionTok{return}\NormalTok{(}\FunctionTok{table}\NormalTok{(x\_out)}\SpecialCharTok{/}\NormalTok{it)}
\NormalTok{\}}

\FunctionTok{get\_trans\_prob}\NormalTok{(}\DecValTok{50}\NormalTok{,}\DecValTok{100}\NormalTok{,}\DecValTok{20}\NormalTok{)}
\end{Highlighting}
\end{Shaded}

\begin{verbatim}
## x_out
##    49    50    51 
## 0.050 0.545 0.405
\end{verbatim}

\begin{Shaded}
\begin{Highlighting}[]
\CommentTok{\# Changing x0 }
\FunctionTok{get\_trans\_prob}\NormalTok{(}\DecValTok{40}\NormalTok{,}\DecValTok{100}\NormalTok{,}\DecValTok{20}\NormalTok{)}
\end{Highlighting}
\end{Shaded}

\begin{verbatim}
## x_out
##    39    40    41 
## 0.041 0.400 0.559
\end{verbatim}

\begin{Shaded}
\begin{Highlighting}[]
\FunctionTok{get\_trans\_prob}\NormalTok{(}\DecValTok{60}\NormalTok{,}\DecValTok{100}\NormalTok{,}\DecValTok{20}\NormalTok{)}
\end{Highlighting}
\end{Shaded}

\begin{verbatim}
## x_out
##    59    60    61 
## 0.044 0.575 0.381
\end{verbatim}

\begin{Shaded}
\begin{Highlighting}[]
\FunctionTok{get\_trans\_prob}\NormalTok{(}\DecValTok{80}\NormalTok{,}\DecValTok{100}\NormalTok{,}\DecValTok{20}\NormalTok{)}
\end{Highlighting}
\end{Shaded}

\begin{verbatim}
## x_out
##    79    80    81 
## 0.035 0.773 0.192
\end{verbatim}

\begin{Shaded}
\begin{Highlighting}[]
\CommentTok{\# Changing N}
\FunctionTok{get\_trans\_prob}\NormalTok{(}\DecValTok{50}\NormalTok{,}\DecValTok{80}\NormalTok{,}\DecValTok{20}\NormalTok{)}
\end{Highlighting}
\end{Shaded}

\begin{verbatim}
## x_out
##    49    50    51 
## 0.045 0.629 0.326
\end{verbatim}

\begin{Shaded}
\begin{Highlighting}[]
\FunctionTok{get\_trans\_prob}\NormalTok{(}\DecValTok{50}\NormalTok{,}\DecValTok{120}\NormalTok{,}\DecValTok{20}\NormalTok{)}
\end{Highlighting}
\end{Shaded}

\begin{verbatim}
## x_out
##    49    50    51 
## 0.056 0.404 0.540
\end{verbatim}

\begin{Shaded}
\begin{Highlighting}[]
\FunctionTok{get\_trans\_prob}\NormalTok{(}\DecValTok{50}\NormalTok{,}\DecValTok{160}\NormalTok{,}\DecValTok{20}\NormalTok{)}
\end{Highlighting}
\end{Shaded}

\begin{verbatim}
## x_out
##    49    50    51 
## 0.035 0.344 0.621
\end{verbatim}

\begin{Shaded}
\begin{Highlighting}[]
\CommentTok{\# Changing s}
\FunctionTok{get\_trans\_prob}\NormalTok{(}\DecValTok{50}\NormalTok{,}\DecValTok{100}\NormalTok{,}\DecValTok{15}\NormalTok{)}
\end{Highlighting}
\end{Shaded}

\begin{verbatim}
## x_out
##    49    50    51 
## 0.064 0.498 0.438
\end{verbatim}

\begin{Shaded}
\begin{Highlighting}[]
\FunctionTok{get\_trans\_prob}\NormalTok{(}\DecValTok{50}\NormalTok{,}\DecValTok{100}\NormalTok{,}\DecValTok{10}\NormalTok{)}
\end{Highlighting}
\end{Shaded}

\begin{verbatim}
## x_out
##    49    50    51 
## 0.071 0.518 0.411
\end{verbatim}

\begin{Shaded}
\begin{Highlighting}[]
\FunctionTok{get\_trans\_prob}\NormalTok{(}\DecValTok{50}\NormalTok{,}\DecValTok{100}\NormalTok{,}\DecValTok{1}\NormalTok{)}
\end{Highlighting}
\end{Shaded}

\begin{verbatim}
## x_out
##    49    50    51 
## 0.338 0.497 0.165
\end{verbatim}

The primary objective of the get\_trans\_prob function is to simulate a
random walk process characterized by two sequential steps, both
influenced by probabilistic transitions. \newline When increasing x0 the
probability of assigning y to 1 grows, because the probability for
getting 1 from the vector c(0,1) is calculated by x/N. For the second
step, the probability of getting a 1 assigned to the variable z is
calculated by (s\emph{x)/(s}x+(N-1) and therefore for having a weight of
e.g.~20 the probability of getting 1 grows with increasing x0. So the
probabilities for both steps are higher for getting 1 with increasing x0
and therefore the most probable new assignments for x are in the first
step x = x - 1 and x = x + 1 in the second step, which can also be
observed in the distribution of the three possible x values in the table
as the number of middle values grows with increasing x0. \newline When
increasing N the probability of assigning the variable y to 1 decreases,
because of the formula x/N. Therefore it is more likely to assign y to 0
and x not be subtracted by 1. In the second step the probability for
assigning z to 1 is high due to the strong effect of the weight variable
s (as discussed above). \newline When decreasing s the probability for
assigning the variable z to 1 also decreases and therefore it is less
likely to add 1 to x. This results in a higher frequency of the lowest
of the three values.

\hypertarget{d-bonus-how-can-this-process-be-applied-to-population-genetics-and-evolution}{%
\subsection{d) Bonus: How can this process be applied to population
genetics and
evolution?}\label{d-bonus-how-can-this-process-be-applied-to-population-genetics-and-evolution}}

The described process can be applied to population genetics and
evolution by representing the initial allele frequency (x0) of a gene in
a population (N). The parameter (s) influences transitions, akin to
factors like selection, migration, or mutation rates. The number of
iterations (it) simulates generations, with each step reflecting genetic
events like drift or selection. This process allows to study how
evolutionary forces shape allele frequencies over time, offering
insights into genetic diversity and population dynamics.

\hypertarget{task-3-cpg-islands-and-hmms}{%
\section{Task 3 CpG islands and
HMMs}\label{task-3-cpg-islands-and-hmms}}

\hypertarget{a-compare-the-entries-of-cg-and-cg-and-interprete-the-result.}{%
\subsection{\texorpdfstring{a) Compare the entries of \texttt{C−G−} and
\texttt{C+G+} and interprete the
result.}{a) Compare the entries of C−G− and C+G+ and interprete the result.}}\label{a-compare-the-entries-of-cg-and-cg-and-interprete-the-result.}}

The probability of getting from a C to a G in a non CpG island is 0.21\%
while the likelihood of getting from a C to a G in a CpG island is
21.91\% (factor 100), which makes sense given the fact that CpG islands
are characterized by a high number CpG pairs.

\hypertarget{b-how-does-the-emission-matrix-look-like-what-are-the-observables-and-the-emission}{%
\subsection{b) How does the emission matrix look like? What are the
observables and the
emission}\label{b-how-does-the-emission-matrix-look-like-what-are-the-observables-and-the-emission}}

probabilities from the hidden states? (1P)

\begin{Shaded}
\begin{Highlighting}[]
\NormalTok{B }\OtherTok{\textless{}{-}} \FunctionTok{matrix}\NormalTok{(}\FunctionTok{c}\NormalTok{(}\DecValTok{1}\NormalTok{,}\DecValTok{0}\NormalTok{,}\DecValTok{0}\NormalTok{,}\DecValTok{0}\NormalTok{,}\DecValTok{0}\NormalTok{,}\DecValTok{1}\NormalTok{,}\DecValTok{0}\NormalTok{,}\DecValTok{0}\NormalTok{,}\DecValTok{0}\NormalTok{,}\DecValTok{0}\NormalTok{,}\DecValTok{1}\NormalTok{,}\DecValTok{0}\NormalTok{,}\DecValTok{0}\NormalTok{,}\DecValTok{0}\NormalTok{,}\DecValTok{0}\NormalTok{,}\DecValTok{1}\NormalTok{,}\DecValTok{1}\NormalTok{,}\DecValTok{0}\NormalTok{,}\DecValTok{0}\NormalTok{,}\DecValTok{0}\NormalTok{,}\DecValTok{0}\NormalTok{,}\DecValTok{1}\NormalTok{,}\DecValTok{0}\NormalTok{,}\DecValTok{0}\NormalTok{,}\DecValTok{0}\NormalTok{,}\DecValTok{0}\NormalTok{,}\DecValTok{1}\NormalTok{,}\DecValTok{0}\NormalTok{,}\DecValTok{0}\NormalTok{,}\DecValTok{0}\NormalTok{,}\DecValTok{0}\NormalTok{,}\DecValTok{1}\NormalTok{),}\AttributeTok{nrow=}\DecValTok{8}\NormalTok{,}\AttributeTok{byrow=}\NormalTok{T)}
\FunctionTok{rownames}\NormalTok{(B) }\OtherTok{\textless{}{-}} \FunctionTok{c}\NormalTok{(}\StringTok{"A{-}"}\NormalTok{,}\StringTok{"C{-}"}\NormalTok{,}\StringTok{"G{-}"}\NormalTok{,}\StringTok{"T{-}"}\NormalTok{,}\StringTok{"A+"}\NormalTok{,}\StringTok{"C+"}\NormalTok{,}\StringTok{"G+"}\NormalTok{,}\StringTok{"T+"}\NormalTok{)}
\FunctionTok{colnames}\NormalTok{(B) }\OtherTok{\textless{}{-}} \FunctionTok{c}\NormalTok{(}\StringTok{"A"}\NormalTok{,}\StringTok{"C"}\NormalTok{,}\StringTok{"G"}\NormalTok{,}\StringTok{"T"}\NormalTok{)}
\NormalTok{B}
\end{Highlighting}
\end{Shaded}

\begin{verbatim}
##    A C G T
## A- 1 0 0 0
## C- 0 1 0 0
## G- 0 0 1 0
## T- 0 0 0 1
## A+ 1 0 0 0
## C+ 0 1 0 0
## G+ 0 0 1 0
## T+ 0 0 0 1
\end{verbatim}

The observables are the A, C, T, G bases without the island status. The
emission probabilities from the hidden states has to be 100\% for a
corresponding base regardless of its island status and therefore the
other probabilities in the row have to be 0.

\hypertarget{c-use-the-transition-matrix-to-calculate-the-probability-that-the-following-sequence-was-sampled-from-a-cpg-island-and-the-probability-that-it-was-sampled-from-a-non-island-region.-tgacgacgaa.-compare-the-probabilities.}{%
\subsection{c) Use the transition matrix to calculate the probability,
that the following sequence was sampled from a CpG-island and the
probability, that it was sampled from a non-island region. TGACGACGAA.
Compare the
probabilities.}\label{c-use-the-transition-matrix-to-calculate-the-probability-that-the-following-sequence-was-sampled-from-a-cpg-island-and-the-probability-that-it-was-sampled-from-a-non-island-region.-tgacgacgaa.-compare-the-probabilities.}}

\begin{Shaded}
\begin{Highlighting}[]
\NormalTok{prob\_island }\OtherTok{\textless{}{-}} \FloatTok{0.2910} \SpecialCharTok{*} \FloatTok{0.2557} \SpecialCharTok{*} \FloatTok{0.1892} \SpecialCharTok{*}  \FloatTok{0.2191} \SpecialCharTok{*} \FloatTok{0.2557} \SpecialCharTok{*} \FloatTok{0.1892} \SpecialCharTok{*} \FloatTok{0.2191} \SpecialCharTok{*} \FloatTok{0.2557} \SpecialCharTok{*} \FloatTok{0.3190}
\NormalTok{prob\_non\_island }\OtherTok{\textless{}{-}} \FloatTok{0.2158} \SpecialCharTok{*} \FloatTok{0.3654} \SpecialCharTok{*} \FloatTok{0.1629} \SpecialCharTok{*}  \FloatTok{0.0211} \SpecialCharTok{*} \FloatTok{0.3654} \SpecialCharTok{*}  \FloatTok{0.1629} \SpecialCharTok{*} \FloatTok{0.0211} \SpecialCharTok{*} \FloatTok{0.3654} \SpecialCharTok{*} \FloatTok{0.3921}

\NormalTok{prob\_island }\SpecialCharTok{\textgreater{}}\NormalTok{ prob\_non\_island}
\end{Highlighting}
\end{Shaded}

\begin{verbatim}
## [1] TRUE
\end{verbatim}

The probability that the sequence is derived from a CpG-island is higher
than the probability that it is derived from a non-island region.

\begin{enumerate}
\def\labelenumi{\alph{enumi})}
\setcounter{enumi}{3}
\tightlist
\item
  Bonus: Use the R-script on Ilias to determine the most likely sequence
  of hidden paths, that 'produce' the sequence which is also available
  on Ilias. Run the whole script, have a look at the output and answer
  the following question: What is the benefit of an HMM-approach
  compared to the standard CG-content measurement? In general: In which
  scenarios should you rather use the HMM than 'local' measurements?
  \includegraphics{CompBio_Homework_SCHWALL_ELIAS_files/figure-latex/unnamed-chunk-12-1.pdf}
  In general, the benefit of an HMM approach over a standard CG-content
  measurement lies in its ability to capture dependencies between
  consecutive nucleotides and model complex patterns in the sequence.
  HMMs can provide more nuanced information about the underlying hidden
  states (CpG island or non-island) based on observed sequences, making
  them suitable for scenarios where local measurements (like CG-content
  alone) may not capture the underlying structure effectively. HMMs are
  particularly useful when there is a need to account for the context
  and dependencies between adjacent elements in the sequence.
\end{enumerate}

\end{document}
